\subsection{Interface}
	These are the server side scripts that will be used across the website.

	\emph{Constants}
	\begin{itemize}
		\item Header
		\begin{itemize}
			\item session\_start();
			\item Doctype
			\item meta tags
			\begin{itemize}
				\item description
				\item Keywords
				\item authir
				\item content type
				\item CSS links
				\item Font link
				\item validation.js
				\item \$title
			\end{itemize}
			\item Header/Div
			\begin{itemize}
				\item Logo
			\end{itemize}
		\end{itemize}

		\item Nav
		\begin{itemize}
			\item Nav
			\item include`breadcrumbs';
		\end{itemize}
		\item Footer 
		\begin{itemize}
			\item Site map link 
			\item Icon
		\end{itemize}
			
		\item config.php
		\begin{itemize}
			%%\$host = \"\";
			%%\$port \= \"\";
			%%\$user \= \"\";
			%%\$password \= \"\";
			%%\$dbname \= "";
			%%mysql \= new mysql(\$...)
			%%die statements
			\item Address of Server
			\item Admin Password
		\end{itemize}

		\item Index
		\begin{itemize}
			\item \$title
			\item include `header.php';
			\item include `nav.php';
			\item text
			\item include `simple\_search.php';
			\item text
			\begin{itemize}
				\item app link
			\end{itemize}
			\item include `footer.php';
		\end{itemize}	

		\item plants
		\begin{itemize}
			\item include `config.php';
			\item \$title
			\item include `header.php';
			\item include `nav.php';
			\item add\_plant\_record link
			\item include `advanced\_search.php';
			\item include `plant\_db\_declare.php';
			\item include `plant\_sorting.php';
			\item include `footer.php';
		\end{itemize}
		\item add\_plants\_record
		\begin{itemize}
			\item include `config.php';
			\item \$title
			\item include `header.php';
			\item include `nav.php';
			\item form(input)
			\item if no image uploaded, default image else uploaded image
			\item include `upload\_image.php';
			\item include `save\_record.php';
			\item a href `plants.php' cancel
			\item include `footer.php';
		\end{itemize}
		
		\item plants\_record
		\begin{itemize}
			\item include `config.php';
			\item \$title
			\item include `header.php';
			\item include `nav.php';
			\item \$specific record details
			\item a href `edit\_plant\_record.php' edit
			\item a href `delete\_plant\_record.php' remove
			\begin{itemize}
				\item delete prompt js
			\end{itemize}
			\item individual record photo or default image
			\item a href `plants.php' where to find
			\begin{itemize}
				\item (plant\_map.js)
			\end{itemize}
			\item include `footer.php';
		\end{itemize}

		\item edit\_record
		\begin{itemize}
			\item include `config.php';
			\item \$title
			\item include `header.php';
			\item include `nav.php';
			\item include `breadcrumbs.php';
			\item \$individual record
			\item a href `edit\_plant\_record.php' edit
			\item a href `delete\_plant\_record.php' remove
			\item \$individual record photo
			\item a href `plants.php' where to find
			\begin{itemize}
				\item (plant\_map.js)
			\end{itemize}
			\item include `footer.php';
		\end{itemize}
	\end{itemize}

\subsection{Detailed design}

	\subsubsection{config.php}
		config will open a link to the mysql database at the start of each new session, all changes to records and searches are reliant on this script.
		\begin{verbatim}
			$conn = mysql_connect(host=<localhost> 
			port=3306 dbname=<nameofdatabase> 
			user=<uid> password=<password>);
		\end{verbatim}
		The script will also include a few lines closing the link to the database at the end of the session as not to produce more connections to the database than needed.

	\subsubsection{plant\_sorting.php}
		The plant data can be sorted in multiple different ways there will be a default sort using SORT\_REGULAR however the options of SORT\_STRING and SORT\_NUMERIC will also be available as required. The script will provide the user with an option of how they wish to sort the data.

	\subsubsection{simple\_search.php}
		The simple search script will search the database for a string matching the users input, all entries in the database containing that string will be returned. It will require a connection to the database from config.php, if there is no connection then the script will return an error message to notify the user of a problem with the connection to the database.

	\subsubsection{advanced\_search.php}
		The script for advanced search will allow for searches by string, ID, data type, location and user. The user should select their search type before inputting, the results shall be returned unless no result is found or there is an issue with the connection to the database in which case an error message shall be returned to the user. 

	\subsubsection{edit\_plant\_record.php}
		This script will take data input from the client side to form an update in an object oriented way. The update will set the field to be updated to the new data. The script will query if the connection to the server is valid before committing the changes, it will also return a confirmation to the user that the record has been updated or that there was a problem connecting to the server (they have timed out) so they can try again.

	\subsubsection{delete\_plant\_record.php}
		The script for deleting a record will be very straight forward, using an object oriented approach the script will request for the record to be deleted by using its ID. Much like the edit this script will return a confirmation to the user that the record has successfully been deleted or that there was a problem with the connection to the server.

	\subsubsection{upload\_image.php}
		This script should prompt the user to choose an image file to upload first, when an image file has been selected an upload function will collect the data about the file where it can be validated to check it matches the requirements we have set (eg: file type and size). The image file, if compatible, will be uploaded and the script will return a success message, otherwise a message will be returned to the user explaining which parameter is incorrect or if there is an issue with the connection to the server.

	\subsubsection{regex validation}
		validation will be done on the client side using javascript and regular expressions. Doing this allows us to use an onchange method when the user is inputting data with fixed parameters, such as what characters can be used, without putting extra load on the server.

	\subsubsection{Index.php}
		The index script will be the home page for the web site, allowing users to read about the Botany Project, navigate the web site, search for a plant and download the app. There will be a header at the top of the page containing the Botany Group Project title and an image of the logo. The banner will be customised using CSS. Underneath there will be a navigation bar using the nav.php script. Below this there will be two text boxes which will be achieved using the echo function and will be customised to have a white background using CSS. Between the texts there will be a search bar controlled by simple\_search.php. At the bottom the footer.php will be included. A background image will be used for the homepage.

	\subsubsection{Header.php}
		The header will be responsible for creating a session for the user. The header script will be responsible used to declare the doctype and meta tags such as description, keywords, content type, CSS and font links, validation.js and title. 

	\subsubsection{footer.php}
		The footer will contain the site map link allowing users to access all the pages of the website. There will also be an image of the logo in the centre of footer.

	\subsubsection{nav.php}
		The navigation bar provides an easy reference for the contents of the web site and enables the user to navigate the web site conveniently. This will be achieved by attaching a div class to the navigation text with a href attribute, linking it to the corresponding pages. CSS will be used to style the navigation bar and links accordingly with the Web user interface design. Underneath there will be breadcrumbs included using the breadcrumbs script.

	\subsubsection{Edit/add record}
		Upon selecting an individual plant from the database, the plant along with its attributes and an image will be displayed by pulling it from the database using SQL. The user can find where to locate the plant by clicking the ‘Where to find’ button which will run the plant\_map.js script, loading Google maps javascript API. The user can then edit the plant’s information by clicking the edit button supplied, running the edit\_plant\_record.php script, loading the Edit Record page. This page will use the config.php script to form a database connection that will insert a table, ready to retrieve and print the selected fields of the plant chosen by the user. This will be achieved using pg\_query; selecting the required fields and echoing a table. A while loop will be used to read and  pull data from the database. Users can then edit plant information by filling in the form with the attributes assigned to a plant. The form is to be validated using javascript, identifying the required fields. If a required field is left blank, an alert will pop up requesting the user to fill the field in. The image can also be updated by clicking the supplied ‘Upload Image’ submit button, running the upload\_image.php script. Once the form has been edited by the user, they can then save the record using the Save button linked to the save\_record.php script. This will update the database with the new edited record. Users will also be given the option to add new plant records to the database. This page will be the same as the Edit Record page but with a blank form for the user to fill in.

	\subsubsection{delete record}
		Users are given the option to delete a specific plant record. A ‘remove’ button linked to the delete\_plant\_record.php script will be supplied to do this.

	\subsubsection{delete confirmation pop up}
		If a user clicks the remove button, a javascript alert pop up will be displayed confirming the deletion of a record. This has been included to avoid accidental deletions.

	\subsubsection{config.php}
		The config.php file will be used to establish the connection to the mysql database using the servers API at the start of a new session.
		\begin{verbatim}
			$CONFIG=array
			api="file location"
			session="where the session is stored"
		\end{verbatim}
		The config.php will need to be be included in all php scripts.


	\subsubsection{authenticate.php}
		The authenticate.php script will be used to authenticate an admin user on the site. Who through entering a correct password will have access to certain features of the site that regular users will not have. Such as deletion of reserves and specimens. The authentication will be checked to the servers API. If a correct password is entered then the user will see a message confirming it. If an incorrect password is used then a different message will be see saying that a wrong password has been used.


	\subsubsection{delete\_curl.php}
		This script is used to delete specific specimens from the servers database. This feature will only become available once a user has entered a correct password to show that they are an admin user. PHP CURL is used so that the website can communicate with the server.


	\subsubsection{edit\_specimens.php}
		This script is similar to the delete script. It uses PHP CURL to communicate with the server. This function is also only available to admin users.


	\subsubsection{footer.php}
		This script will be included on every page. It will contain a small website logo in the centre of the footer. 


	\subsubsection{get\_reseve.php}
		This will be used to call all the data about the reserves stored on the server. PHP CURL will be used to communicate with the server and to decode the JSON that the API will be using so that it can be displayed on the reserves page.

	\subsubsection{header.php}
		The header will be responsible for creating a session and storing it for the users. The header script will be responsible for declaring the doctype and meta tags such as description, keywords, content type, CSS and font links, validation as well as JAVASCRIPT and the title. 

	\subsubsection{img\_curl.php}
		This is used to find get the specimen and scene photo from the database. PHP CURL is again used to communicate with th server.

	\subsubsection{logout.php}
		This script is used to logout an admin user who has entered a correct password to the site. It will end the session that is being used and display a message telling the user that they have been logout of the site.

	\subsubsection{map.php}
		The contains the JAVASCRPIT to be able to view the plants location within a Google maps pop up. It will use the latitude and longitude variables of each specimen to give an accurate location.

	\subsubsection{specimens\_curl.php}
		This will be used to access the information and each individual specimen. CURL is again used to access the server.

	\subsubsection{resdelete\_curl.php}
		This will work in a very similar way to the delete\_curl.php script but will be used for deleting whole reserves instead of just individual specimens. The user will need the be logged in for this function to work. 

	\subsubsection{reserves\_curl.php}
		Much like the record\_curl.php it will be used to get information form the server, but information about the reserves rather then the specimen details.

	\subsubsection{specimens\_search.php}
		This will be used to search for certain specimens within the specimen page on the website. Users will be able to search by Species Name, Location Name and The User Name of people using the site.

	\subsubsection{filter.php}
		This will also be used on the plants page on the website and will allow users to order by Species Name, Location, User Name, Date and Abundance and allow results to be show in ascending or descending order.

\begin{landscape}
    \begin{figure}
        \centering
        \includegraphics[scale=1]{web/webComponentDiagram.png}
        \caption{Web component Diagram}
        \label{fig:webComponentDiagram}
    \end{figure}
\end{landscape}


