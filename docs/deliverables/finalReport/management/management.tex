Our group were able to deliver all three sections of the project, the android app, the database and the web site. 

For the functional requirements we met FR1, FR2, FR3, FR4, FR6, FR7, FR8 and FR9. We do not meet FR5 fully do to the feature of being able to delete a whole visit. The other 2 features have been completed as species can be deleted as well as edited. For the External Interface Requirement I believe both are intuitive to the average computer user. Both platforms will reflect user input within a second as well as the systems running on the appropriate platforms the android app was tested on multiple android devices and the site optimized for multiple browsers, meeting requirement PR1 and PR2. For DC1 we used java for the application building the application using android studio. The website uses PHP and connects to a MYSQL database. DC2 has been completed for the site with data being implemented in to the database.

Extra functionality was added on the site such as the ability to add a specimen sighting on the website and being able to edit and delete. We use AJAX to search results by Latin name.

With all the work now completed I am happy to say that we delivered a good product, if there was more time there is defiantly room for improvement in certain areas and usability of the different areas but I believe these are small fixes that could be fixed with an extra day of programming. The project plan is a good description of what the team did and I believe we kept to it pretty closely as well as the project design which was a very useful document when it came to linking all the different sections together.